Actuarial modeling is the name for a set of techniques used in the insurance industry. These models are composed of equations that represent the functioning of insurance companies, accounting for the probabilities of the events covered by policies and the costs each event presents to the company. They help companies decide which policies to take and to set premiums based on the projected claims that they will have to pay. They are important because insurance companies use them to keep the companies solvent; models predict the funds that companies will have to pay out, so they know how much money they have to take in to cover their costs.

Insurance companies are organizations that allow policyholders to share risks with each other. The company takes in payments, called premiums, in exchange for the guarantee that it will give money to the policyholder of some specified event occurs. In effect, all of the policyholders are splitting the cost of the events that occur in each period so that no one will have to pay for the entire expense.

An actuary is a person who works for an insurance company and ensures that it charges sufficient premiums to cover overhead costs and the claims that policyholders file. Actuaries use scientific approaches that combine probability theory, economic theory and other disciplines. They use behavioral assumptions derived from these theories to create systems of equations that represent the events that happen in the real world. This practice is called actuarial modeling.

The two basic types of models used in actuarial modeling are deterministic models and stochastic models. Deterministic models are the simpler of the two, and they were the first to be used. They use estimates of probabilities for each event, and they predict the number of events that will actually happen based on these estimates. Stochastic models allow for more randomness, but they require more computational power. A computer simulates the events over a period hundreds or thousands of times, and based on the outcomes of its simulations, it predicts how many events will happen.

The type of model used is of little importance if the actuary does not have good information about the events he is predicting. In actuarial modeling, the probability of each event and the equations that describe people’s behavior are crucial to the success of the model. Actuaries constantly revise models so they yield better predictions.
