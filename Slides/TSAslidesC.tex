
%===================================================%
\begin{frame}
\frametitle{Time Series Analysis}
\textbf{PART C : ANALYSIS OF CYCLICAL VARIATIONS}
Annual time series values represent the effects of only the trend and cyclical components, because the
seasonal and irregular components are defined as short-run influences. Therefore, for annual data the
cyclical component can be identified as being the component that would remain in the data after the influence of
the trend component is removed. This removal is accomplished by dividing each of the observed values by the
associated trend value, as follows:
Y
YT
¼
T   C
T
¼ C (16:9)
\end{frame}
%===================================================%
\begin{frame}
\frametitle{Time Series Analysis}
\begin{itemize}
\item The ratio in (16.9) is multiplied by 100 so that the mean cyclical relative will be 100.0. A cyclical relative
of 100 would indicate the absence of any cyclical influence on the annual time series value. See Problem 16.2.
\item In order to aid in the interpretation of cyclical relatives, a cycle chart which portrays the cyclical relatives
according to year is often prepared. 
item The peaks and troughs associated with the cyclical component of the time
series can be made more apparent by the construction of such a chart. See Problem 16.3.
\end{itemize}
\end{frame}
%===================================================%
\begin{frame}
\frametitle{Time Series Analysis}
\textbf{PART D  MEASUREMENT OF SEASONAL VARIATIONS}
\begin{itemize}
\item The influence of the seasonal component on time series values is identified by determining the seasonal
index number associated with each month (or quarter) of the year. 
\item The arithmetic mean of all 12 monthly index
numbers (or four quarterly index numbers) is 100. 
\item The identification of positive and negative seasonal
influences is important for production and inventory planning.
\end{itemize}
\end{frame}
%===================================================%
\begin{frame}
\frametitle{Time Series Analysis}
EXAMPLE 2. 
\begin{itemize}
\item An index number of 110 associated with a given month indicates that the time series values for that month
have averaged 10 percent higher than for other months because of some positive seasonal factor. 
\item For instance, the unit sales
of men’s shavers might be 10 percent higher in June as compared with other months because of Father’s Day.
\item The procedure most frequently used to determine seasonal index numbers is the ratio-to-moving-average
method.
\item By this method, the ratio of each monthly value to the moving average centered at that month is first
determined. 
\end{itemize}
\end{frame}
%===================================================%
\begin{frame}
\frametitle{Time Series Analysis}
Because a moving average based on monthly (or quarterly) data for an entire year would average
out the seasonal and irregular fluctuations, but not the longer-term trend and cyclical influences, the ratio of a
monthly (or quarterly) value to a moving average can be represented symbolically by
Y
Moving average
¼
T   C   S   I
T   C
¼ S   I (16:10)

\end{frame}
%===================================================%
\begin{frame}
\frametitle{Time Series Analysis}
\begin{itemize}
\item The second step in the ratio-to-moving-average method is to average out the irregular component.
\item  This is
typically done by listing the several ratios applicable to the same month (or quarter) for the several years,
eliminating the highest and lowest values, and computing the mean of the remaining ratios.
\item  The resulting mean
is called a modified mean, because of the elimination of the two extreme values.
\item The final step in the ratio-to-moving-average method is to adjust the modified mean ratios by a
correction factor so that the sum of the 12 monthly ratios is 1,200 (or 400 for four quarterly ratios). See
Problem 16.4.
\end{itemize}
\end{frame}
