\documentclass{beamer}

\usepackage{framed}
\usepackage{amsmath}
\begin{document}
%===================================================%
\begin{frame}
\frametitle{Time Series Analysis}
\huge
\[ \mbox{ Time Series Analysis
and Business
Forecasting} \]

\end{frame}
%===================================================%
\begin{frame}
\frametitle{Time Series Analysis}
\textbf{PART A: THE CLASSICAL TIME SERIES MODEL}
\begin{itemize}
\item A time series is a set of observed values, such as production or sales data, for a sequentially ordered series
of time periods. 
\item Examples of such data are sales of a particular product for a series of months and the number of
workers employed in a particular industry for a series of years. 
\item A time series is portrayed graphically by a line
graph, with the time periods represented on the horizontal axis and time series values
represented on the vertical axis.
\end{itemize}
%% LINE GRAPH
\end{frame}
%===================================================%
\begin{frame}
\frametitle{Time Series Analysis}
\textbf{EXAMPLE 1.}
\begin{itemize}
\item  Figure 16-1 is a line graph that portrays the annual dollar sales for a graphics software company (fictional)
that was incorporated in 1990. 
\item As can be observed, a peak in annual sales was achieved in 1995, followed by two years of
declining sales that culminated in the trough in 1997, which was then followed by increasing levels of sales during the
final three years of the reported time series values.
\end{itemize}
\end{frame}
%===================================================%
\begin{frame}
\frametitle{Time Series Analysis}
\begin{itemize}
\item Time series analysis is the procedure by which the time-related factors that influence the values observed in
the time series are identified and segregated. 
\item Once identified, they can be used to aid in the interpretation of
historical time series values and to forecast future time series values. 
\end{itemize}
\end{frame}
%===================================================%
\begin{frame}
\frametitle{Time Series Analysis}
The classical approach to time series
analysis identifies four such influences, or, components:
\begin{itemize}
\item[(1)] Trend (T): The general long-term movement in the time series values (Y) over an extended period of years.
\item[(2)] Cyclical fluctuations (C): Recurring up and down movements with respect to trend that have a duration of
several years.
\item[(3)] Seasonal variations (S): Up and down movements with respect to trend that are completed within a year and
recur annually. Such variations typically are identified on the basis of monthly or quarterly data.
\item[(4)] Irregular variations (I): The erratic variations from trend that cannot be ascribed to the cyclical or seasonal
influences.
\end{frame}
%===================================================%
\begin{frame}
\frametitle{Time Series Analysis}
The model underlying classical time series analysis is based on the assumption that for any designated
period in the time series the value of the variable is determined by the four components defined above, and,
furthermore, that the components have amultiplicative relationship. Thus, where Y represents the observed time
series value,
\[Y = T \times  C \times  S \times  I \]
The model represented by (16.1) is used as the basis for separating the influences of the various components
that affect time series values, as described in the following sections of this chapter.

\end{frame}
%===================================================%
\begin{frame}
\frametitle{Time Series Analysis}
\textbf{Part B: TREND ANALYSIS}
\begin{itemize}
\item Because trend analysis is concerned with the long-term direction of movement in the time series, such
analysis generally is performed using annual data. 
\item Typically, at least 15 or 20 years of data should be used, so
that cyclical movements, involving several years duration are not taken to be indicative of the overall trend of
the time series values.
\end{itemize}
\end{frame}
%===================================================%
\begin{frame}
\frametitle{Time Series Analysis}
\begin{itemize}
\item The method of least squares is the most frequent basis used for identifying the trend
component of the time series by determining the equation for the best-fitting trend line. 
\item Note that statistically
speaking, a trend line is not a regression line, since the dependent variable Y is not a random variable, but,
rather, a series of historical values. 
\item Further, there can be only one historical value for any given time period (not
a distribution of values) and the values associated with adjoining time periods are likely to be dependent, rather
than independent. 
\end{itemize}
\end{frame}
%===================================================%
\begin{frame}
\frametitle{Time Series Analysis}
Nevertheless, the least-squares method is a convenient basis for determining the trend
component of a time series. When the long-term increase or decrease appears to follow a linear trend, the
equation for the trend line values, with X representing the year, is
YT ¼ b0 þ b1X (16:2)
As explained in Section 14.3, the b0 in (16.2) represents the point of intersection of the trend line with the Y
axis, whereas the b1 represents the slope of the trend line. Where X is the year and Y is the observed time series
value, the formulas for determining the values of b0 and b1 for the linear trend equation are
%% EQUATION
\end{frame}
%===================================================%
\begin{frame}
\frametitle{Time Series Analysis}
See Problem 16.1 for the determination of a linear trend equation.
In the case of nonlinear trend, a type of trend curve often found to be useful is the exponential trend curve.
A typical exponential trend curve is one that reflects a constant rate of growth during a period of years, as might
CHAP. 16] TIME SERIES ANALYSIS AND BUSINESS FORECASTING 297
apply to the sales of personal computers during the 1980s. See Fig. 16-2(a). An exponential curve is so named
because the independent variable X is the exponent of b1 in the general equation:
YT ¼ b0 bX
1 (16:5)
where b0 ¼ value of YT in Year 0
b1 ¼ rate of growth (e.g., 1:10 ¼ 10% rate of growth)
\end{frame}
%===================================================%
\begin{frame}
\frametitle{Time Series Analysis}
\begin{itemize}
\item Taking the logarithm of both sides of (16.5) results in a linear logarithmic trend equation,
log YT ¼ log b0 þ X log b1 (16:6)
\item The advantage of the transformation into logarithms is that the linear equation for trend analysis can be
applied to the logs of the values when the time series follows an exponential curve.
\item  The forecasted log values
for YT can then be reconverted to the original measurement units by taking the antilog of the values. We do not
demonstrate such analysis in this book.
\end{itemize}
\end{frame}
%===================================================%
\begin{frame}
\frametitle{Time Series Analysis}
\begin{itemize}
\item Many time series for the sales of a particular product can be observed to include three stages: an
introductory stage of slow growth in sales, a middle stage of rapid sales increases, and a final stage of slow
growth as market saturation is reached.
\item For some products, such as structural steel, the complete set of three
stages may encompass many years. 
\item For other products, such as citizen band radios, the stage of saturation may
be reached relatively quickly. 
\item One particular trend curve that includes the three stages just described is the
S-shaped Gompertz curve, as portrayed in Fig. 16-2(b). 
\end{itemize}
\end{frame}
%===================================================%
\begin{frame}
\frametitle{Time Series Analysis}
\textbf{Gompertz Trend}\\
An equation that can be used to fit the Gompertz trend
curve is
YT ¼ b0 b(b2) X
1 (16:7)
The values of b0, b1, and b2 are determined by first taking the logarithm of both sides of the equation, as
follows:
log YT ¼ log b0 þ ( log b1)bX
2 (16:8)

\end{frame}
%===================================================%
\begin{frame}
\frametitle{Time Series Analysis}
Finally, the values to form the trend curve are calculated by taking the antilog of the values calculated
by Formula (16.8). The details of such calculations are included in specialized books on time series
analysis.

\end{frame}
